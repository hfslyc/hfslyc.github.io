%%%%%%%%%%%%%%%%%%%%%%%%%%%%%%%%%%%%%%%%%
% "ModernCV" CV and Cover Letter
% LaTeX Template
% Version 1.1 (9/12/12)
%
% This template has been downloaded from:
% http://www.LaTeXTemplates.com
%
% Original author:
% Xavier Danaux (xdanaux@gmail.com)
%
% License:
% CC BY-NC-SA 3.0 (http://creativecommons.org/licenses/by-nc-sa/3.0/)
%
% Important note:
% This template requires the moderncv.cls and .sty files to be in the same 
% directory as this .tex file. These files provide the resume style and themes 
% used for structuring the document.
%
%%%%%%%%%%%%%%%%%%%%%%%%%%%%%%%%%%%%%%%%%

%----------------------------------------------------------------------------------------
%	PACKAGES AND OTHER DOCUMENT CONFIGURATIONS
%----------------------------------------------------------------------------------------

\documentclass[11pt,letterpaper,sans]{moderncv} % Font sizes: 10, 11, or 12; paper sizes: a4paper, letterpaper, a5paper, legalpaper, executivepaper or landscape; font families: sans or roman

\moderncvstyle{classic} % CV theme - options include: 'casual' (default), 'classic', 'oldstyle' and 'banking'
\moderncvcolor{blue} % CV color - options include: 'blue' (default), 'orange', 'green', 'red', 'purple', 'grey' and 'black'

\usepackage{lipsum} % Used for inserting dummy 'Lorem ipsum' text into the template

\usepackage[scale=0.82]{geometry} % Reduce document margins
\setlength{\hintscolumnwidth}{3.7cm} % Uncomment to change the width of the dates column
\setlength{\makecvtitlenamewidth}{10cm} % For the 'classic' style, uncomment to adjust the width of the space allocated to your name

\graphicspath{{../}{image/}}
\newcommand{\ul}[1]{\underline{\smash{#1}}}
\def\red#1{\textcolor{red}{#1}}
\def\blue#1{\textcolor{blue}{#1}}
\def\oral#1{\textcolor{red}{\textbf{#1}}}

%----------------------------------------------------------------------------------------
%	NAME AND CONTACT INFORMATION SECTION
%----------------------------------------------------------------------------------------

\firstname{Wei-Chih} % Your first name
\familyname{Hung\\(Wayne)} % Your last name

% All information in this block is optional, comment out any lines you don't need
\title{Curriculum Vitae}
\address{311 Science and Engineering Building 2}{UC Merced, CA 95343}
%\mobile{209-777-2216}
\phone{+1-213-453-3980}
%\fax{}
\email{whung8@ucmerced.edu}
\homepage{hfslyc.github.io/}{https://hfslyc.github.io/} 
% The first argument is the url for the clickable link, the second argument is the url displayed in the template - this allows special characters to be displayed such as the tilde in this example
%\extrainfo{additional information}
%\photo[70pt][0.4pt]{pictures/picture} % The first bracket is the picture height, the second is the thickness of the frame around the picture (0pt for no frame)
%\quote{"A witty and playful quotation" - John Smith}

%----------------------------------------------------------------------------------------

\begin{document}

\makecvtitle % Print the CV title

%--------------------------------------------------------------------------------------
% EDUCATION SECTION
%--------------------------------------------------------------------------------------
\vspace{-10mm}
\section{Education}
%%%%%%%%%%%%%%%%%%%%%%%%%%%%%%%%%%%%%%%%%%%%%%%%%%%%%%%%%%%%%%%%%%%%%%%%%%%%%%%%
\cventry{2016--Present}
{Ph.D. Student}
{\href{http://ucmerced.edu/}{University of California, Merced}}{CA}{USA}
{
Electrical Engineering and Computer Science \\
Vision and Learning Lab \infolink{http://vllab.ucmerced.edu}{link}
}
%%%%%%%%%%%%%%%%%%%%%%%%%%%%%%%%%%%%%%%%%%%%%%%%%%%%%%%%%%%%%%%%%%%%%%%%%%%%%%%%

\cventry{2014--2016}
{Masters of Science}
{\href{http://www.usc.edu//}{University of Southern California}}{CA}{USA}
{Electrical Engineering\\
Media Communication Lab \infolink{http://mcl.usc.edu/}{link}}

%%%%%%%%%%%%%%%%%%%%%%%%%%%%%%%%%%%%%%%%%%%%%%%%%%%%%%%%%%%%%%%%%%%%%%%%%%%%%%%%

\cventry{2011--2013}
{Masters of Science}
{\href{http://www.ntu.edu.tw/english/}{National Taiwan University}}{Taipei}{Taiwan}
{Communication Engineering}

%%%%%%%%%%%%%%%%%%%%%%%%%%%%%%%%%%%%%%%%%%%%%%%%%%%%%%%%%%%%%%%%%%%%%%%%%%%%%%%%

\cventry{2007--2011}
{Bachelor of Science}
{\href{http://www.ntu.edu.tw/english/}{National Taiwan University}}{Taipei}{Taiwan}
{Electrical Engineering}


%--------------------------------------------------------------------------------------
% RESEARCH INTERESTS
%--------------------------------------------------------------------------------------
\section{Research Interests}
\cvitem{}{Computer Vision, Machine Learning}



%--------------------------------------------------------------------------------------
% PUBLICATION SECTION
%--------------------------------------------------------------------------------------

\section{Publications  \small{ (\infolink{https://scholar.google.com/citations?hl=en&btnA=1&user=AjaDLjYAAAAJ}{Google Scholar profile})}}

%\cventry{Submit to CVPR 2017}
%{Scene Parsing with Global Context Encoding}
%{}{}{}
%{
%	\ul{Wei-Chih Hung}, Yi-Hsuan Tsai, Xiaohui Shen, Zhe Lin, Kalyan Sunkavalli, Xin Lu, and Ming-Hsuan Yang \\
%	Submit to IEEE Conference on Computer Vision and Pattern Recognition, 2017
%}
%
%
%\cventry{Submit to CVPR 2017}
%{Instance Semantic Segmentation by Learning Pairwise Affinity}
%{}{}{}
%{
%	\ul{Wei-Chih Hung}, Xiao Bian, and Ser-Nam Lim \\
%	Submit to IEEE Conference on Computer Vision and Pattern Recognition, 2017
%}


\cventry{ECCV 2016}
{Unsupervised Visual Representation Learning by Graph-Based Consistent Constraints}
{}{}{}
{
	Dong Li, \ul{Wei-Chih Hung}, Jia-Bin Huang, Shengjin Wang, Narendra Ahuja, Ming-Hsuan Yang\\
	IEEE Conference on Computer Vision and Pattern Recognition, 2016
	{\infolink{https://drive.google.com/open?id=0BynEQyOSGRoSOW1ILU5NMFgwTUk}{paper}}
	{\infolink{https://sites.google.com/site/lidonggg930/feature-learning}{project}}
}



\cventry{GlobalComm 2013}
{Iterative decoding for uncompressed wireless video transmission}
{}{}{}
{
	Wei-Ting Lin*, \ul{Wei-Chih Hung*} (*indicates equal contribution), Kuan-Yu Lin, Ping-Cheng Yeh\\
	IEEE Global Communications Conference, 2013
}

\cventry{WCNC 2012}
{Dynamic source-channel rate-distortion control under time-varying complexity constraint for wireless video transmission}
{}{}{}
{
	Tsu-Hao Kuo*, Po-Hsuan Chen*, \ul{Wei-Chih Hung}, Chih-Yu Huang, Chia-han Lee, and Ping-Cheng Yeh\\
	IEEE Wireless Communications and Networking Conference, 2012
}



%--------------------------------------------------------------------------------------
%	RESEARCH EXPERIENCE SECTION
%--------------------------------------------------------------------------------------

\section{Research Experience}

\cventry{May. 2016 - Aug. 2016}
%
{Computer Vision Lab, GE Research, Niskayuna, NY}
{}{}{}
{
	\textit{Research Intern with Xiao Bian and Ser Nam Lim}
	%
	\begin{itemize}
		\item \textbf{Instance Semantic Segmentation by Learning Pairwise Affinity} \\
		This work aims to solve the instance segmentation by learning from the semantic affinity between pixels based on a fully convolutional network.
	\end{itemize}
}


\cventry{Aug. 2016 -- Present}
%
{Vision and Learning Lab, EECS, University of California, Merced}
{}{}{}
{
\textit{Graduate Research Assistant with Prof. Ming-Hsuan Yang}
%
\begin{itemize}
	%
	%
	\item \textbf{Scene Parsing with Global Context Encoding (Collaborate with Adobe Research})\\
	This work aims to solve scene parsing by incorporating the scene category information with the Siamese network and improving the parsing results through both parametric and non-parametric methods.
	%
	%
	\item \textbf{Unsupervised Learning by Graph-Based Consistent Constraints}\\
	This work aims to perform deep unsupervised learning by leveraging the graph consistency between images.
\end{itemize}
}




\cventry{Jul. 2014 - Jul. 2016}
%
{Media Communication Lab, University of Southern California}
{}{}{}
{
	\textit{Research Assistant with Prof. C.-C. Jay Kuo}
	%
	\begin{itemize}
		\item \textbf{Object Verification for Pedestrian Detection} \\
		This work aims at handling intra-class variation in the pedestrian detection problem under low image quality using figure-ground segmentation and contour straddling measure as a second-stage classifier.
		%
		\item \textbf{Data Driven Indoor Scene 3D Layout Understanding} \\
		This work aims to approach indoor scene understanding by using geometry cues with structure learning algorithms.
		\item \textbf{Remote Mentoring System based on Smart Glasses for Aircraft Maintenance} \\
		Developed a remote mentoring system with smart glasses with optimized streaming quality for Korean Air and United Technology. 
	\end{itemize}
}



\cventry{Jul. 2011 - Jun. 2013}
%
{Multimedia Communication Lab, National Taiwan University, Taipei, Taiwan}
{}{}{}
{
	\textit{Research Assistant with Prof. Ping-Cheng Yeh}
	%
	\begin{itemize}
		\item \textbf{Iterative 3D-MRF based Decoder for Uncompressed Wireless Video Transmission} \\
		This work aims to improve the wireless video transmission quality by introducing a 3-dimensional (spatio-temporal) Markov random field (MRF) model to formulate the natural redundancy of video sequences . 
		\item \textbf{Joint Research on 3GPP LTE and LTE-Advanced Physical Layer with HTC Cooperation} \\
		This project aims to contribute to the latest 4G protocol meeting by proposing a new MIMO precoder codebook by interpolating multiple feedback precoder matrix index (PMI) using geodesic field Interpolation.
	\end{itemize}
}


\cventry{Jul. 2010 - Jun. 2011}
%
{Wireless Communication Lab, Academia Sinica, Taipei, Taiwan}
{}{}{}
{
	\textit{Research Assistant with Prof. Chia-Han Lee}
	%
	\begin{itemize}
		\item \textbf{Joint Source-Channel Rate-Distortion Control under Dynamic Complexity Constraint for Wireless Video Transmission} \\
		This work proposes an online algorithm searching for H.264 parameters to reach sub-optimal distortion in real-time.
		\item \textbf{Software-defined Radio based Wireless H.264 Video Streaming System} \\
		This project develops a software-defined radio (SDR) based wireless H.264 video streaming system over Universal Software Radio Peripheral (USRP) and GNU Radio.
	\end{itemize}
}

\cventry{Jul. 2011 - Sep. 2011}
%
{Qualcomm, Taipei, Taiwan}
{}{}{}
{
	\textit{Software Intern, Multimedia Group}
	%
	\begin{itemize}
		\item \textbf{Mobile GPU Analysis} \\
		This project develops internal tools using opengl-es on the most advanced mobile platform with mobile GPU team.
	\end{itemize}
}




%--------------------------------------------------------------------------------------
%	TEACHING EXPERIENCE SECTION
%--------------------------------------------------------------------------------------

\section{Teaching Experience}

\cventry{Aug. 2016 -- Present}
%
{EECS, University of California, Merced}
{}{}{}
{
\begin{itemize}
	\item CSE 020 Introduction to Computing [Java Programming] (Fall 2016)
\end{itemize}
}


%--------------------------------------------------------------------------------------
%	AWARDS SECTION
%--------------------------------------------------------------------------------------
\section{Awards}

\cventry{Feb. 2011}
{First Prize}{}
{Nvidia Parallel Computing Contest 2011}{}{Develop a real-time indoor sound simulation system with CUDA.}

\cventry{Oct. 2010}
{Undergraduate Student Research Grant}{}
{Academia Sinica, Taipei, Taiwan}{}{Lead a research project on wireless video transmission system.}


%--------------------------------------------------------------------------------------
%	ACADEMIC SERVICES SECTION
%--------------------------------------------------------------------------------------
%\section{Academic Services}
%
%\cvitem{Conference Reviewer}
%{
%European Conference on Computer Vision (\textbf{ECCV 2016}) \newline
%Asian Conference on Computer Vision (\textbf{ACCV 2016}) \newline
%Neural Information Processing Systems (\textbf{NIPS 2016}) \newline
%Pacific Conference on Computer Graphics and Applications (\textbf{PG 2016})
%}
%
%\cvitem{Journal Reviewer}
%{
%	Computer Vision and Image Understanding (\textbf{CVIU 2016})
%}
%--------------------------------------------------------------------------------------
%	SKILLS SECTION
%--------------------------------------------------------------------------------------

\section{Technical Skills}

\cvitem{Programming}{C/C++, Python, Java}
\cvitem{Toolbox / Software}{Caffe, MATLAB, OpenCV, CUDA}


%--------------------------------------------------------------------------------------
%	REFERENCES SECTION
%--------------------------------------------------------------------------------------

\section{References}

\cventry{Ph.D. Advisor}
{Ming-Hsuan Yang}
{Associate Professor}{University of California, Merced}{}
{
	\emailsymbol\emaillink{mhyang@ucmerced.edu}
	\infolink{http://faculty.ucmerced.edu/mhyang/}{homepage}
}

\cventry{Research Mentor}
{Ser Nam Lim}
{Lab Manager, Computer Vision Lab}{GE Global Research, Niskayuna, NY}{}
{
	\emailsymbol\emaillink{limser@ge.com}
	\infolink{https://www.linkedin.com/in/sernam}{homepage}
}

\cventry{M.S. Advisor}
{C.-C. Jay Kuo}
{Dean's Professor}{University of Southern California, Los Angeles}{}
{
	\emailsymbol\emaillink{cckuo@sipi.usc.edu}
	\infolink{http://mcl.usc.edu/people/cckuo/}{homepage}
}


\cventry{M.S. Advisor}
{Ping-Cheng Yeh}
{Professor}{National Taiwan University, Taiwan}{}
{
	\emailsymbol\emaillink{pcyeh@ntu.edu.tw}
	\infolink{http://ee.ntu.edu.tw/en/article/teacher.person/jobSN/1/navSN/356/webSN/376/teacherSN/109}{homepage}
}




\end{document}